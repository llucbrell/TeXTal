%\usepackage{listings}
%\usepackage[utf8]{inputenc}
%\usepackage[utf8x]{inputenc}
%\usepackage[T1]{fontenc}
%\usepackage{lmodern}
\usepackage[T1]{fontenc}
\usepackage{listingsutf8}
\usepackage{color}
\usepackage[T1]{fontenc}
\usepackage{parcolumns}
% change the name of listings caption pre
\renewcommand{\lstlistingname}{C\'odigo}% Listing -> Algorithm
\renewcommand{\lstlistlistingname}{Lista de  \lstlistingname s}% List of Listings -> List of Algori

\definecolor{mygreen}{rgb}{0,0.6,0}
\definecolor{myblack}{rgb}{0,0,0}
\definecolor{mygray}{rgb}{0.55,0.5,0.5}
\definecolor{mymauve}{rgb}{0.58,0,0.82}
\definecolor{codegreen}{rgb}{0,0.6,0}
\definecolor{codegray}{rgb}{0.5,0.5,0.5}
\definecolor{codepurple}{rgb}{0.58,0,0.82}
\definecolor{backcolour}{rgb}{0.97,0.97,0.97}

%%%%%%%%%%%%%%%%%%%%%%%%%%%%%%%%%%%%%%%%
%   características del código
%%%%%%%%%%%%%%%%%%%%%%%%%%%%%%%%%%%%%%%%
%  backgroundcolor=\color{white},   % choose the background color; you must add \usepackage{color} or \usepackage{xcolor}; should come as last argument
%  basicstyle=\footnotesize,        % the size of the fonts that are used for the code
%  breakatwhitespace=false,         % sets if automatic breaks should only happen at whitespace
%  breaklines=true,                 % sets automatic line breaking
%  captionpos=b,                    % sets the caption-position to bottom
%  commentstyle=\color{mygreen},    % comment style
%  deletekeywords={...},            % if you want to delete keywords from the given language
%  escapeinside={\%*}{*)},          % if you want to add LaTeX within your code
%  extendedchars=true,              % lets you use non-ASCII characters; for 8-bits encodings only, does not work with UTF-8
%  frame=single,	                   % adds a frame around the code
%  keepspaces=true,                 % keeps spaces in text, useful for keeping indentation of code (possibly needs columns=flexible)
%  keywordstyle=\color{blue},       % keyword style
%  language=Octave,                 % the language of the code
%  morekeywords={*,...},            % if you want to add more keywords to the set
%  numbers=none,                    % where to put the line-numbers; possible values are (none, left, right)
%  numbersep=5pt,                   % how far the line-numbers are from the code
%  numberstyle=\tiny\color{mygray}, % the style that is used for the line-numbers
%  rulecolor=\color{black},         % if not set, the frame-color may be changed on line-breaks within not-black text (e.g. comments (green here))
%  showspaces=false,                % show spaces everywhere adding particular underscores; it overrides 'showstringspaces'
%  showstringspaces=false,          % underline spaces within strings only
%  showtabs=false,                  % show tabs within strings adding particular underscores
%  stepnumber=2,                    % the step between two line-numbers. If it's 1, each line will be numbered
%  stringstyle=\color{mymauve},     % string literal style
%  tabsize=2,	                   % sets default tabsize to 2 spaces
%  title=\lstname                   % show the filename of files included with \lstinputlisting; also try caption instead of title

\lstdefinestyle{no-color}{
	flexiblecolumns=true,	
	keepspaces=false,
	backgroundcolor=\color{white},  
	keywordstyle=\bfseries\color{black}, 
	stringstyle=\color{black},
	commentstyle=\color{black},
	basicstyle=\footnotesize,
	tabsize=2,	
	morekeywords={}, 
	deletekeywords={}, 
	frame=none,
	captionpos=b,
	numbers=none,
	showtabs=false, 
	showspaces=false,
	showstringspaces=false,
	aboveskip=1cm,	
	belowskip=1cm
	}
\lstdefinestyle{no-color-line}{
	flexiblecolumns=true,	
	keepspaces=false,
	backgroundcolor=\color{white},  
	keywordstyle=\bfseries\color{black}, 
	stringstyle=\color{black},
	commentstyle=\color{black},
	basicstyle=\footnotesize,
	tabsize=2,	
	morekeywords={}, 
	deletekeywords={}, 
	frame=simple,
	captionpos=t,
	numbers=none,
	showtabs=false, 
	showspaces=false,
	showstringspaces=false,
	aboveskip=1cm,	
	belowskip=1cm
	}
\lstdefinestyle{no-color-frame}{
	flexiblecolumns=true,	
	keepspaces=false,
	backgroundcolor=\color{white},  
	keywordstyle=\bfseries\color{black}, 
	stringstyle=\color{black},
	commentstyle=\color{black},
	basicstyle=\footnotesize,
	tabsize=2,	
	morekeywords={}, 
	deletekeywords={}, 
	frame=single,
	captionpos=b,
	numbers=none,
	showtabs=false, 
	showspaces=false,
	showstringspaces=false,
	aboveskip=1cm,	
	belowskip=1cm
	}
\lstdefinestyle{colorA}{
	flexiblecolumns=true,	
	keepspaces=false,
	backgroundcolor=\color{white},  
	keywordstyle=\color{blue},
	stringstyle=\color{mymauve},
	commentstyle=\color{mygreen},
	basicstyle=\footnotesize,
	tabsize=2,	
	morekeywords={}, 
	deletekeywords={}, 
	frame=none,
	captionpos=b,
	numbers=none,
	showtabs=false, 
	showspaces=false,
	showstringspaces=false,
	aboveskip=1cm,	
	belowskip=1cm
	}
\lstdefinestyle{colorA-frameround}{
	flexiblecolumns=true,	
	keepspaces=false,
	backgroundcolor=\color{white},  
	keywordstyle=\color{blue},
	stringstyle=\color{mymauve},
	commentstyle=\color{mygreen},
	basicstyle=\tiny,
	tabsize=2,	
	morekeywords={}, 
	deletekeywords={}, 
	frame=trBL,
	frameround=fttt,
	captionpos=b,
	numbers=none,
	showtabs=false, 
	showspaces=false,
	showstringspaces=false,
	aboveskip=1cm,	
	belowskip=1cm
	}
\lstdefinestyle{colorB}{
	flexiblecolumns=true,	
	keepspaces=false,
	backgroundcolor=\color{backcolour},  
	keywordstyle=\color{magenta},
	stringstyle=\color{codepurple},
	commentstyle=\color{green},
	basicstyle=\footnotesize,
	tabsize=2,	
	morekeywords={}, 
	deletekeywords={}, 
	frame=none,
	captionpos=b,
	numbers=none,
	showtabs=false, 
	showspaces=false,
	showstringspaces=false,
	aboveskip=1cm,	
	belowskip=1cm
	}
\lstdefinestyle{colorBB}{
	flexiblecolumns=true,	
	keepspaces=false,
	backgroundcolor=\color{backcolour},   
    	commentstyle=\color{codegreen},
        keywordstyle=\color{magenta},
	stringstyle=\color{codepurple},
	basicstyle=\tiny,
	tabsize=2,	
	morekeywords={}, 
	deletekeywords={}, 
	frame=none,
	captionpos=b,
	numbers=none,
	showtabs=false, 
	showspaces=false,
	showstringspaces=false,
	aboveskip=1cm,	
	belowskip=1cm
	}
\lstdefinestyle{color}{
%  backgroundcolor=\color{white},   % choose the background color; you must add \usepackage{color} or \usepackage{xcolor}; should come as last argument
  basicstyle=\footnotesize,        % the size of the fonts that are used for the code
  breakatwhitespace=false,         % sets if automatic breaks should only happen at whitespace
  breaklines=true,                 % sets automatic line breaking
  captionpos=b,                    % sets the caption-position to bottom
  commentstyle=\color{mygreen},    % comment style
%  deletekeywords={...},            % if you want to delete keywords from the given language
  escapeinside={\%*}{*)},          % if you want to add LaTeX within your code
  extendedchars=true,              % lets you use non-ASCII characters; for 8-bits encodings only, does not work with UTF-8
%  frame=single,	                   % adds a frame around the code
  keepspaces=true,                 % keeps spaces in text, useful for keeping indentation of code (possibly needs columns=flexible)
  keywordstyle=\color{blue},       % keyword style
  language=Octave,                 % the language of the code
  morekeywords={*,...},            % if you want to add more keywords to the set
  numbers=none,                    % where to put the line-numbers; possible values are (none, left, right)
  numbersep=5pt,                   % how far the line-numbers are from the code
  numberstyle=\tiny\color{mygray}, % the style that is used for the line-numbers
  rulecolor=\color{black},         % if not set, the frame-color may be changed on line-breaks within not-black text (e.g. comments (green here))
  showspaces=false,                % show spaces everywhere adding particular underscores; it overrides 'showstringspaces'
  showstringspaces=false,          % underline spaces within strings only
  showtabs=false,                  % show tabs within strings adding particular underscores
%  stepnumber=2,                    % the step between two line-numbers. If it's 1, each line will be numbered
  stringstyle=\color{mymauve},     % string literal style
  tabsize=2,	                   % sets default tabsize to 2 spaces
%  title=\lstname                   % show the filename of files included with \lstinputlisting; also try caption instead of title
}

\lstdefinestyle{color2}{
	backgroundcolor=\color{backcolour},   
    	commentstyle=\color{codegreen},
        keywordstyle=\color{magenta},
%	numberstyle=\tiny\color{codegray},
	stringstyle=\color{codepurple},
	basicstyle=\tiny,
	breakatwhitespace=false,         
	breaklines=true,                 
	captionpos=b,                    
	keepspaces=true,                 
%	numbers=left,                    
%	numbersep=5pt,                  
	showspaces=false,                
	showstringspaces=false,
	frame=single,
	showtabs=false,
	tabsize=2
}
\lstdefinestyle{colorBB}{
	backgroundcolor=\color{backcolour},   
    	commentstyle=\color{codegreen},
        keywordstyle=\color{magenta},
	stringstyle=\color{codepurple},
	basicstyle=\tiny,
	breakatwhitespace=false,         
	breaklines=true,                 
	captionpos=b,                    
	keepspaces=true,                 
	showspaces=false,                
	showstringspaces=false,
	frame=single,
	showtabs=false,
	tabsize=2,
  	flexiblecolumns=true,
	aboveskip=1cm,
	belowskip=1cm
}

\lstdefinestyle{color3}{
    	commentstyle=\color{codegreen},
        keywordstyle=\color{magenta},
	numberstyle=\tiny\color{codegray},
	stringstyle=\color{codepurple},
	%basicstyle=\footnotesize,
	breakatwhitespace=false,         
	breaklines=true,                 
	captionpos=b,                    
	keepspaces=true,                 
	numbers=left,                    
	numbersep=5pt,                  
	showspaces=true,                
	showstringspaces=true,
	showtabs=false,
	tabsize=2
	}

\lstdefinestyle{colorBA}{
  backgroundcolor=\color{white},   % choose the background color; you must add \usepackage{color} or \usepackage{xcolor}; should come as last argument
  flexiblecolumns=true,		   % don't overlap characters
%  basicstyle=\footnotesize,        % the size of the fonts that are used for the code
%  breakatwhitespace=false,         % sets if automatic breaks should only happen at whitespace
%  breaklines=true,                 % sets automatic line breaking
  captionpos=b,                    % sets the caption-position to bottom
  commentstyle=\color{mygreen},    % comment style
%  deletekeywords={...},            % if you want to delete keywords from the given language
%  escapeinside={\%*}{*)},          % if you want to add LaTeX within your code
%  extendedchars=true,              % lets you use non-ASCII characters; for 8-bits encodings only, does not work with UTF-8
  frame=none,	                   % adds a frame around the code
%  keepspaces=true,                 % keeps spaces in text, useful for keeping indentation of code (possibly needs columns=flexible)
  keywordstyle=\color{blue},       % keyword style
%  language=Octave,                 % the language of the code
  morekeywords={},            % if you want to add more keywords to the set
  numbers=none,                    % where to put the line-numbers; possible values are (none, left, right)
%  numbersep=5pt,                   % how far the line-numbers are from the code
%  numberstyle=\tiny\color{mygray}, % the style that is used for the line-numbers
%  rulecolor=\color{black},         % if not set, the frame-color may be changed on line-breaks within not-black text (e.g. comments (green here))
%  showspaces=false,                % show spaces everywhere adding particular underscores; it overrides 'showstringspaces'
%  showstringspaces=false,          % underline spaces within strings only
  showtabs=false,                  % show tabs within strings adding particular underscores
%  stepnumber=2,                    % the step between two line-numbers. If it's 1, each line will be numbered
  stringstyle=\color{mymauve},     % string literal style
  tabsize=2,	                   % sets default tabsize to 2 spaces
%  title=\lstname                   % show the filename of files included with \lstinputlisting; also try caption instead of title
  aboveskip=2cm,		  % top margin
  belowskip=2cm			  % bottom margin
}


