\section*{Prólogo}

Una de las habilidades básicas que debe poseer todo estudiante es la agilidad. Agilidad para escribir y editar, texto. Cuando escribe una memoria en un editor como libreoffice, abiword u openoffice, presta atención a las faltas de ortografía y al formato adecuado que le da al texto, a la inserción de las imágenes, código fuente etc.,  Los editores de este estilo, WYSWYG (lo que ves es lo que tienes) tienen la particularidad de mostrar por pantalla exactamente lo que se va a visualizar y lo hacen muy bien. Pero esta característica, a la hora de realizar un escrito de cierta envergadura puede distraer, y más si hay que aportar código informático, imágenes, además de no conocer extensamente el editor. La labor a realizar se va complicando. Para lidiar con este problema, hay múltiples soluciones de carácter personal o técnico, la expresada en esta documentación es solamente una de esas soluciones posibles. 
\section*{Objetivo}

Esta plantilla trata de cubrir un doble objetivo a nivel particular así como también doble a nivel de comunidad. 
\subsection*{Particular}

\textbf{$\alpha$} -Ayudar al autor en la asimilación de conceptos, a la hora de ir aprendiendo a usar \hologo{LaTeX} texto. La práctica ayuda en la asimilación de conceptos.Así, realizando esta plantilla, se trata de conseguir la meta del aprendizaje básico de una herramienta como es \hologo{LaTeX} y \hologo{BibTeX}.

\textbf{$\beta$}  -Preparar una plantilla para cuando se realice el proyecto de fin de módulo. Como estudiante, de programación existe una obligación moral de crear herramientas que nos ayuden a realizar nuestras tareas de manera más eficiente.


\subsection*{Comunidad}

\textbf{$\gamma$} -Apoyar de manera indirecta, mediante la difusión y promoción a la comunidad \hologo{LaTeX}.

\textbf{$\delta$} -Mejorar si es posible a los no doctos, en la introducción en el mundo de este lenguaje.

\section*{¿Dónde encontrar la plantilla?}
Puedes conseguir el código fuente de la plantilla entre los repositorios que encontrarás en la siguiente url:

\begin{center}\url{https://github.com/llucbrell}\end{center}

\section*{Utilización}
Para utilizar esta plantilla, se recomienda leer este archivo así como también el archivo \textbf{README.txt} que se puede encontrar en el directorio raíz del anterior repositorio. Se recomienda descargarse alguna chuleta de comandos básicos para \hologo{LaTeX} y tenerla a mano mientras se escribe la memoria del proyecto. 

\section*{Colabora}
Si quieres puedes colaborar añadiendo nuevos capítulos a este documento a modo de explicación o símplemente nuevo código y subiendo posteriormente al repositorio los cambios usando  GIT. Las colaboraciones son bienvenidas. 
